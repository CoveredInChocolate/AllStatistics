
\newpage
\section{Chapter 2 - Random Variables}

\subsection*{Exercises}

%%%%%%%%%%%%%%%%%%%%%%%%%%%%%%%%%%%%%%%%%%%%%%%%%%%%%%%%%%%%%%%%%%%%%%%%%%%%%%%
\textbf{2.1}\\  % PDF page 43
\textbf{Claim}: $\P(X = x) = F(x^+) - F(x^-)$. (Discrete)\\
\textsc{Proof}. By definition of the CDF:
$$
F(x^+) = \lim_{z\downarrow x}F(z) = \lim_{z\downarrow x}\P(X\leq z),\qquad
F(x^-) = \lim_{y\uparrow x}F(y) = \lim_{y\uparrow x}\P(X\leq y)
$$
(so $y<x$ and $y\ra x$, and $x<z$ and $x\leftarrow z$). By the right continuous property,
we can deduce that $z > y$ and we can set $z = x$ and $y = x-1$.
$$
\P(X\leq x^+) = \P(X\leq x) = \P(X = x) + \P(X \leq x-1),\quad
\P(X\leq x^-) = \P(X\leq x-1)
$$
So:
\begin{align*}
    \P(X = x) &= \P(X = x) + \P(X\leq x-1) - \P(X\leq x-1) \\
    &= \P(X\leq x) - \P(X\leq x-1) \\
    &= \P(X\leq x^+) - \P(X\leq x^-) \\
    &= F(x^+) - F(x^-)
    \tag*{\qed}
\end{align*}

\bigskip\noindent
%%%%%%%%%%%%%%%%%%%%%%%%%%%%%%%%%%%%%%%%%%%%%%%%%%%%%%%%%%%%%%%%%%%%%%%%%%%%%%%
\textbf{2.2}\\  % PDF page 44
Let $X$ be such that $\P(X = 2) = \P(X = 3) = 1/10$ and $\P(X = 5) = 8/10$.
Here is a plot of the CDF.
\begin{center}
    \begin{tikzpicture}
\begin{axis}[
    clip=false,
    jump mark left,
    ymin=0,ymax=1,
    xmin=0, xmax=8,
    every axis plot/.style={very thick},
    discontinuous,
    table/create on use/cumulative distribution/.style={
        create col/expr={\pgfmathaccuma + \thisrow{f(x)}}   
    }
]
\addplot [blue] table [y=cumulative distribution]{
x f(x)
0 0
2 1/10
3 1/10
5 8/10
8 0
};
\end{axis}
\end{tikzpicture}
\end{center}
By reading the plot, we can see that:
$$
\P(2 < X\leq 4.8) = F(4.8) - F(2) = 2/10 - 1/10 = 1/10
$$
$$
\P(2\leq X\leq 4.8) = F(4.8) = 2/10
$$

\bigskip\noindent
%%%%%%%%%%%%%%%%%%%%%%%%%%%%%%%%%%%%%%%%%%%%%%%%%%%%%%%%%%%%%%%%%%%%%%%%%%%%%%%
\textbf{2.3}\\  % PDF page 44
\textbf{Lemma 2.15} Let $F$ be the CDF for a random variable $X$. Then:
\begin{enumerate}
    \item $\P(X = x) = F(x) - F(x^-)$
    \item $\P(x < X\leq y) = F(y) - F(x)$
    \item $\P(X > x) = 1 - F(x)$
    \item If $X$ is continuous, then
$$
F(b) - F(a) = \P(a < X < b) = \P(a \leq X < b) = \P(a < X\leq b) = \P(a\leq X\leq b)
$$
\end{enumerate}
\textsc{Proof}. We will prove each statement in turn. (1.) was proved in exercise \textbf{2.1}.
Doing (3) first, since we need it to prove (2).

\medskip\noindent(3) By definition of complements of sets $A = \{X > x\}$ means $A^c = \{X\leq x\}$,
and it follows that:
$$
\P(X > x) = \P(A) = 1 - \P(A^c) = 1 - \P(X\leq x) = 1 - F(x).
$$

\medskip\noindent(2) Assume $x < y$. We will need that $\{X > x\}\cup\{X\leq y\} = \Omega$,
and we will also use Lemma 1.6 (in reverse).
\begin{align*}
    \P(x < X\leq y) &= \P(\{X > x\}\cap\{X\leq y\}) \\
    &= \P(X > x) + \P(X\leq y) - \P(\{X > x\}\cup\{X\leq y\}) \\
    &= 1 - F(x) + F(y) - 1 \\
    &= F(y) - F(x)
\end{align*}

\medskip\noindent(4) Similar argument for all cases, so will just do one.
We just need to turn the inequalities into
strict inequalities. For continuous random variables, pointwise probabilities are 0.
Again, we will need to use $\{X > a\}\cup\{X < b\} = \Omega$.

Define $A := \{a \leq X\}$ and $B := \{X < b\}$.
First, we make the following observation:
\begin{align*}
    \P(A) &= \P(\{a \leq X\}) \\
    &= \P(\{a = X\}\cup\{a < X\}) \\
    &= \P(\{a = X\}) + \P(\{a < X\}) + \P(\{a = X\}\cap\{a < X\}) \\
    &= 0 + \P(A') + 0 \\
    &= \P(A')
\end{align*}
where $A' = \{a < X\}$. We get 0 for the pointwise probability, since this is continuous,
and we get 0 because the sets are disjoint. We have shown that $\P(A) = \P(A')$ and can use
this to conclude the proof.
\begin{align*}
    \P(a \leq X < b) &= \P(\{a\leq X\}\cap\{X < b\})\\
    &= \P(A\cap B) \\
    &= \P(A) + \P(B) + \P(A\cup B) \\
    &= \P(A) + \P(B) + \P(\Omega) \\
    &= \P(A') + \P(B) + \P(A'\cup B) \\
    &= \P(A'\cap B) \\
    &= \P(a < X < b)
    \tag*{\qed}
    %&= \P\bigg(\Big[\{X = a\}\cup \{a < X\}\Big]\cap \{X < b\}\bigg)
\end{align*}

\bigskip\noindent
%%%%%%%%%%%%%%%%%%%%%%%%%%%%%%%%%%%%%%%%%%%%%%%%%%%%%%%%%%%%%%%%%%%%%%%%%%%%%%%
\textbf{2.4}\\  % PDF page 44
$X$ has the probability density (PDF):
$$
f_X(x) = 
\left\{
    \begin{matrix}
        1/4 & 0<x<1 \\
        3/8 & 3<x<5 \\
        0 & \text{otherwise}
    \end{matrix}
\right.
$$
Plot of the PDF:
\begin{center}
    \begin{tikzpicture}
        \begin{axis}[
            axis lines = left,
            ymin = -0.002,
            ymax = 0.5,
            xlabel = $x$,
            ylabel = {$f_X(x)$},
        ]
        %Section 1
        \addplot [
            domain=0:1, 
            samples=10, 
            color=blue,
            style=ultra thick,
        ]
        {1/4};
        %Section 2
        \addplot [
            domain=1:3, 
            samples=10, 
            color=blue,
            style=ultra thick,
        ]
        {0};
        %Section 3
        \addplot [
            domain=3:5, 
            samples=10, 
            color=blue,
            style=ultra thick,
        ]
        {3/8};
        %Section 4
        \addplot [
            domain=5:7, 
            samples=10, 
            color=blue,
            style=ultra thick,
        ]
        {0};
        %Vertical lines
        \addplot +[mark=none, color=blue, style=dashed] coordinates {(1, -0) (1, 1/4)};
        \addplot +[mark=none, color=blue, style=dashed] coordinates {(3, -0) (3, 3/8)};
        \addplot +[mark=none, color=blue, style=dashed] coordinates {(5, -0) (5, 3/8)};
        \end{axis}
        \end{tikzpicture}
\end{center}
From the relatively simple structure, we can easily determine the area under the graph:
$$
A = (1)\left(\frac{1}{4}\right) + (2)\left(\frac{3}{8}\right) = \frac{2}{8} + \frac{6}{8} = 1
$$
(a) Finding the CDF by integrating the PDF. We will split up the integral in several parts.
First for the case when $y\in(0,1)$:
$$
F_X(y) = \int_{-\infty}^y f_X(t)dt = \frac{1}{4}\int_0^y 1dt  
= \frac{1}{4}\Big[t\Big]_0^y 
= \frac{y}{4}
$$
When $y=1$ we have $F_X(1) = 1/4$.
Next, we must consider the case $y\in(1,3)$. Here the PDF is 0, so it doesn't increase.
It remains constant at 1/4 (since the CDF doesn't decrease).
$$
F_X(y) = \frac{1}{4}
$$
Next is the case $y\in(3,5)$. Consider the intermediary integral:
$$
I_1= \int_3^y\frac{3}{8}dt
= \frac{3}{8}\Big[t\Big]_3^y
= \frac{3y - 9}{8}
$$
For values $y\in(3,5)$ we start on 1/4, so the CDF in this region becomes:
$$
F_X(y) = \frac{3y - 9}{8} + \frac{1}{4}
$$
% $$
% F_X(y) = \int_{-\infty}^y f_X(t)dt = \int_0^y \frac{1}{4}dt + \int_3^y\frac{3}{8}dt = I_1 + I_2
% $$
% \begin{align*}
%     I_1 =
%     \int_0^y \frac{1}{4}dt  
%     = \frac{1}{4}\int_0^y 1dt  
%     = \frac{1}{4}\Big[t\Big]_0^y 
%     = \frac{y}{4}
% \end{align*}
% \begin{align*}
%     I_2 = \int_3^y\frac{3}{8}dt 
%     = \frac{3}{8}\int_3^y 1 dt 
%     = \frac{3}{8}\Big[t\Big]_3^y
%     = \frac{3y - 9}{8}
% \end{align*}
% With these results, we can write:
% $$
% F_X(y) =
% \left\{
%     \begin{matrix}
%         \displaystyle \frac{y}{4} & 0 < y < 1 \\
%         \displaystyle \frac{3y - 9}{8} & 3 < y < 5 \\
%     \end{matrix}
% \right.
% $$
% Note that when $y = 5$ we get:
% $$
% F_X(5) = \frac{3(5) - 9}{8} + \frac{(1)}{4} = \frac{6}{8} + \frac{2}{8} = 1
% $$
\newpage\noindent
So, the full expression for the CDF becomes:
$$
F_X(y) =
\left\{
    \begin{matrix}
        \displaystyle y/4 & y\in(0,1) \\
        1/4 & y\in(1,3) \\
        \displaystyle \frac{3y - 9}{8} + \frac{1}{4} & y\in(3,5) \\
        1 & y\geq 5
    \end{matrix}
\right.
$$
Note that when $y = 5$ we get:
$$
F_X(5) = \frac{3(5) - 9}{8} + \frac{1}{4} = \frac{6}{8} + \frac{2}{8} = 1
$$
Plot of the CDF:
\begin{center}
    \begin{tikzpicture}
    \begin{axis}[
        axis lines = left,
        ymin = -0.002,
        ymax = 1.05,
        xlabel = $x$,
        ylabel = {$y$},
    ]
    %Section 1
    \addplot [
        domain=0:1, 
        samples=10, 
        color=blue,
        style=ultra thick,
    ]
    {x/4};
    %Section 2
    \addplot [
        domain=1:3, 
        samples=10, 
        color=blue,
        style=ultra thick,
    ]
    {1/4};
    %Section 3
    \addplot [
        domain=3:5, 
        samples=10, 
        color=blue,
        style=ultra thick,
    ]
    {(3*x - 9)/8 + 1/4};
    %Section 4
    \addplot [
        domain=5:7, 
        samples=10, 
        color=blue,
        style=ultra thick,
    ]
    {1};
    \end{axis}
    \end{tikzpicture}
\end{center}
% (b) Defining $Y = 1/X$ and finding the PDF of $Y$. Following the instructions
% mentioned at equation (2.11) and example 2.46 on page 41. We usually begin by finding the set:
% $$
% A_y = \{x : r(x)\leq y\} = \{x : 1/x\leq y\} = \{x : x\leq 1/y\}.
% $$
% But following the hint we are given, we will consider the following three sets:
% $$
% A_1 = \frac{1}{5} \leq y \leq \frac{1}{3},\quad
% A_2 = \frac{1}{3} \leq y \leq 1,\quad
% A_3 = y\geq 1
% $$
(b) Defining $Y = 1/X$ and finding the PDF of $Y$. Following the hint we are given,
we will consider the following three sets:
$$
A_1 = \frac{1}{5} \leq y \leq \frac{1}{3},\quad
A_2 = \frac{1}{3} \leq y \leq 1,\quad
A_3 = y\geq 1
$$
Where $A_1$ corresponds to $(3, 5)$, $A_2$ to $(1, 3)$
and $A_3$ to $(0,1)$. We can express the CDF for $F_Y(y)$ in terms
of $F_X(x)$: %First, we consider $A_1: y\in(1/5,1/3)$.
\begin{align*}
    F_Y(y) &= \P(Y\leq y) = \P(\frac{1}{X} \leq y) \\
    &= \P(X \geq \frac{1}{y}) \\
    &= 1 - \P(X\leq \frac{1}{y}) \\
    &= 1 - F_X(\frac{1}{y})
\end{align*}
\newpage\noindent
First, we consider $A_1: y\in[1/5,1/3]$, and when we input $1/y$ to
$F_X(\cdot)$, it will be in $(3, 5)$. So:
\begin{align*}
    F_Y(y) &= 1 - F_X(1/y) \\
    &= 1 - \left(\frac{3(\frac{1}{y}) - 9}{8} + \frac{1}{4}\right) \\
    &= 1 - \frac{3 - 9y}{8y} - \frac{1}{4} \\
    &= \frac{3}{4} + \frac{9y - 3}{8y} \\
    &= \frac{15y - 3}{8y}
\end{align*}
Next, we consider $A_2: y\in[1/3,1]$. The input to $F_X(\cdot)$ will be in $(1, 3)$:
\begin{align*}
    F_Y(y) &= 1 - F_X(1/y) \\
    &= 1 - \frac{1}{4} \\
    &= \frac{3}{4}
\end{align*}
Next, we consider $A_3: y\geq 1$. The input to $F_X(\cdot)$ will be in $(0, 1)$:
\begin{align*}
    F_Y(y) &= 1 - F_X(1/y) \\
    &= 1 - \frac{\frac{1}{y}}{4} \\
    &= 1 - \frac{1}{4y}
\end{align*}
Also, whenever $y<1/5$, then $1/y > 5$ which means $F_X(\cdot) = 1$, and so:
$$
F_Y(y) = 1 - F_X(1/y) = 1 - 1 = 0.
$$
This gives a full description of the CDF for $F_Y(y)$.
$$
F_Y(y) =
\left\{
    \begin{matrix}
        0 & y < 1/5 \\
        \rule{0pt}{20pt}\displaystyle \frac{15y - 3}{8y} & 1/5\leq y\leq 1/3 \\
        \rule{0pt}{20pt}\displaystyle \frac{3}{4} & 1/3\leq y \leq 1 \\
        \rule{0pt}{20pt}\displaystyle 1 - \frac{1}{4y} & y\geq 1
    \end{matrix}
\right.
$$

\newpage\noindent
Plot of CDF:
\begin{center}
    \begin{tikzpicture}
    \begin{axis}[
        axis lines = left,
        ymin = -0.002,
        ymax = 1.05,
        xlabel = $y$,
        ylabel = {$F_Y(y)$},
    ]
    %Section 1
    \addplot [
        domain=0:0.2, 
        samples=10, 
        color=blue,
        style=ultra thick,
    ]
    {0};
    %Section 2
    \addplot [
        domain=0.2:0.333, 
        samples=10, 
        color=blue,
        style=ultra thick,
    ]
    {(15*x - 3)/(8*x)};
    %Section 3
    \addplot [
        domain=0.333:1, 
        samples=10, 
        color=blue,
        style=ultra thick,
    ]
    {3/4};
    %Section 4
    \addplot [
        domain=1:2, 
        samples=10, 
        color=blue,
        style=ultra thick,
    ]
    {1 - 1/(4*x)};
    \end{axis}
    \end{tikzpicture}
\end{center}
Finally, we can find the PDF of $Y$. We differentiate each of the parts in
the CDF. When $y\in(1/5, 1/3)$:
\begin{align*}
    \frac{d}{dy}\left(\frac{15y - 3}{8y}\right) &= \frac{3}{8y^2}
\end{align*}
When $y\geq 1$:
\begin{align*}
    \frac{d}{dy}\left(1 - \frac{1}{4y}\right) &= \frac{1}{4y^2}
\end{align*}
(All other parts are constant, so they become 0). This gives us the PDF and its plot:

\medskip\noindent
\begin{minipage}[b]{0.4\textwidth}
    $$
    f_Y(y) =
    \left\{
        \begin{matrix}
            0 & y < 1/5 \\
            \rule{0pt}{20pt}\displaystyle \frac{3}{8y^2} & 1/5\leq y\leq 1/3 \\
            \rule{0pt}{15pt}\displaystyle 0 & 1/3 < y < 1 \\
            \rule{0pt}{20pt}\displaystyle \frac{1}{4y^2} & y\geq 1
        \end{matrix}
    \right.
    $$
    \rule{0pt}{2pt}
    \end{minipage}
\begin{minipage}[c]{0.6\textwidth}
    %Plot of PDF:\\
        \begin{tikzpicture}
        \begin{axis}[
            axis lines = left,
            ymin = -0.002,
            ymax = 10,
            xlabel = $y$,
            ylabel = {$F_Y(y)$},
        ]
        %Section 1
        \addplot [
            domain=0:0.2, 
            samples=10, 
            color=blue,
            style=ultra thick,
        ]
        {0};
        %Section 2
        \addplot [
            domain=0.2:0.333, 
            samples=10, 
            color=blue,
            style=ultra thick,
        ]
        {(3)/(8*x^2)};
        %Section 3
        \addplot [
            domain=0.333:1, 
            samples=10, 
            color=blue,
            style=ultra thick,
        ]
        {0};
        %Section 4
        \addplot [
            domain=1:2, 
            samples=10, 
            color=blue,
            style=ultra thick,
        ]
        {1/(4*x^2)};
        %Vertical lines
        \addplot +[mark=none, color=blue, style=dashed] coordinates {(1/5, -0) (1/5, 9.375)};
        \addplot +[mark=none, color=blue, style=dashed] coordinates {(1/3, -0) (1/3, 3.375)};
        \addplot +[mark=none, color=blue, style=dashed] coordinates {(1, -0) (1, 1/4)};
        \end{axis}
    \end{tikzpicture}
\end{minipage}

\newpage\noindent
\textbf{2.5}\\  % PDF page 44
Let $X$ and $Y$ be discrete RV. $X$ and $Y$ are independent
if and only if $f_{X,Y}(x, y) = f_X(x)f_Y(y)$ for all $x$ and $y$. \\
\textsc{Proof}.\\
$\Rightarrow$) Assume that $X$ and $Y$ are independent. That means that for any $x,y$,
we have
$$
\P(X=x\cap Y=y) = \P(X=x)\P(Y=y)
$$
Starting with the definition of the joint pdf:
\begin{align*}
    f_{X,Y}(x, y) &= \P(X=x, Y=y) \\
    &= \P(X=x\cap Y=y) \\
    &= \P(X=x)\P(Y=y) \\
    &= f_X(x)f_Y(y)
\end{align*}
Which shows that $f_{X,Y}(x, y) = f_X(x)f_Y(y)$ for all $x$ and $y$.

\medskip\noindent
$\Leftarrow$) Assume that $f_{X,Y}(x, y) = f_X(x)f_Y(y)$ for all $x$ and $y$.
By definition:
\begin{align*}
    f_{X,Y}(x, y) &= \P(X=x, Y=y) \\
    &= \P(X=x\cap Y=y)
\end{align*}
And,
\begin{align*}
    f_X(x)f_Y(y) &= \P(X=x)\P(Y=y)
\end{align*}
From our assumption, these are equal, so $\P(X=x\cap Y=y) = \P(X=x)\P(Y=y)$ which shows that
$X$ and $Y$ are independent.\\
By implication both ways, the statement is proved.\qed

\bigskip\noindent
%%%%%%%%%%%%%%%%%%%%%%%%%%%%%%%%%%%%%%%%%%%%%%%%%%%%%%%%%%%%%%%%%%%%%%%%%%%%%%%
\textbf{2.6}\\  % PDF page 45
Let $X$ have distribution $F$ and density $f$, and let $A$ be a subset of the
real line, e.g. $A = (a,b)$ for some $a,b\in\R$ and $a<b$. We have the indicator function
$$
I_A(x) =
\left\{
    \begin{matrix}
        1 & x\in A \\
        0 & x\not\in A
    \end{matrix}
\right.
$$
We will set $Y = I_A(X)$ and find the PDF and CDF of $Y$. The exercise asks for a

\medskip\noindent
probability mass function, but that cannot be correct. Since $X$ has a density $f$, it is
a continuous RV. If $X\sim U(0,1)$ and $A = (0,1)$, then $Y = X$ and it will be a uniform
variable with a continuous distribution.

\medskip\noindent
Since this must be a continuous distribution, what happens if we define $A = \mathbb{Q}\subset\R$?
There will be an infinite number of points in any interval with measure 0. Then we cannot define
a PDF at all... Poorly formulated exercise in my opinion! Skipping for now.

\newpage\noindent
%%%%%%%%%%%%%%%%%%%%%%%%%%%%%%%%%%%%%%%%%%%%%%%%%%%%%%%%%%%%%%%%%%%%%%%%%%%%%%%
\textbf{2.7}\\  % PDF page 45
Let $X$ and $Y$ be independent and suppose that $X,Y\sim U(0,1)$.
For $Z = \min(X, Y)$ we will find the density $f_Z(z)$ for $Z$. Following the hint, we will first
find $\P(Z > z)$.






\begin{comment}

\bigskip\noindent
%%%%%%%%%%%%%%%%%%%%%%%%%%%%%%%%%%%%%%%%%%%%%%%%%%%%%%%%%%%%%%%%%%%%%%%%%%%%%%%
\textbf{2.X}\\  % PDF page 44


\begin{align*}
    A &= B
\end{align*}


\begin{equation*}
    A = B
    \tag*{\qed}
\end{equation*}


\end{comment}