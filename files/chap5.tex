
\newpage
\section{Chapter 5 - Convergence of Random Variables}

\medskip\noindent\textbf{Definition 5.1} Types of convergence\\
Let $X_1,X_2,\ldots$ be a sequence of random variables and let $X$ be another
random variable. Let $F_n$ denote the CDF of $X_n$ and let $F$ denote the CDF of $X$.

\medskip\noindent
1. $X_n$ converges to $X$ in \emph{probability}, $X_n\raP X$, if for all $\epsilon > 0$,
$$
\P(|X_n - X| > \eps)\ra 0
$$
as $n\ra\infty$.

\medskip\noindent
2. $X_n$ converges to $X$ in \emph{distribution}, $X_n\raD X$, if
$$
\lim_{n\ra\infty}F_n(t) = F(t)
$$
at all $t$ for which $F$ is continuous.

\medskip\noindent\textbf{Definition 5.2}\\
$X_n$ converges to $X$ in \emph{quadratic mean} ($L_2$), written $X_n\raQ X$, if
$$
\E[(X_n - X)^2]\ra 0
$$
as $n\ra\infty$.

\medskip\noindent
Relationship in convergence types.
$$
\raQ \imp \raP \imp \raD
$$

\subsection*{Exercises}
%%%%%%%%%%%%%%%%%%%%%%%%%%%%%%%%%%%%%%%%%%%%%%%%%%%%%%%%%%%%%%%%%%%%%%%%%%%%%%%
\textbf{5.1}\\  % PDF page 82
Let $X_1,\ldots, X_n$ be IID with finite mean and variance $\mu$ and $\sigma^2$.
Let $\bar{X}$ be the sample mean and $S_n^2$ be the sample variance.
$$
\bar{X} = \frac{1}{n}\sum_{i=1}^n,
\qquad
S_n^2 = \frac{1}{n-1}\sum_{i=1}^n(X_i - \bar{X})^2
$$

\medskip\noindent(a) Showing that $\E[S_n^2] = \sigma^2$ was done in exercise 3.8.

\medskip\noindent(b) Showing that $S_n^2\raP\sigma^2$. Directly, this means that for any $\eps>0$,
$$
\P(|S_n^2 - \sigma^2| > \eps) = 0 \;\;\text{as}\;\; n\ra\infty.
$$
But we can do it in a different way. Following the hint, we want to rewrite $S_n^2$
so we can apply the law of large numbers.

\medskip\noindent
Doing the calculations on the next page.

\newpage\noindent
\begin{align*}
    S_n^2 &= \frac{1}{n-1}\sum_{i=1}^n(X_i - \bar{X})^2 \\
    &= \frac{n}{n-1}\left(\frac{1}{n}\sum_{i=1}^n\Big[X_i^2 - 2X_i\bar{X} + \bar{X}^2\Big]\right) \\
    \shortintertext{Distributing the sum.}
    &= \frac{n}{n-1}\left(\frac{1}{n}\sum_{i=1}^nX_i^2 - 2\bar{X}\left(\frac{1}{n}\sum_{i=1}^n X_i\right) + \bar{X}^2\right) \\
    &= \frac{n}{n-1}\left(\frac{1}{n}\sum_{i=1}^nX_i^2 - 2\bar{X}^2 + \bar{X}^2\right) \\
    &= \frac{n}{n-1}\left(\frac{1}{n}\sum_{i=1}^nX_i^2 - \bar{X}^2\right) \\
    \shortintertext{WLLN on $\bar{X}^2$ with $g(x) = x^2$.}
    &= \frac{n}{n-1}\left(\frac{1}{n}\sum_{i=1}^nX_i^2 - \mu^2\right) \\
    &= \frac{n}{n-1}\left(\frac{1}{n}\sum_{i=1}^n(X_i - \mu)^2\right)
\end{align*}
The last rewrite is a standard identity. Now, we define $Y_i = X_i - \mu$. Then
$\bar{Y} \ra \E[(X_i - \mu)]$ by the WLLN. By Theorem 5.5(f), we can apply $g(x) = x^2$
and get $\bar{Y}^2 \ra \E[(X_i - \mu)^2] = \V(X_i) = \sigma^2$. Finally, we use the hint
with $c_n = n/(n-1)$ which obviously tends to 1, so we can apply
Theorem 5.5(d) and hence we have proved that $S_n^2 \raP \sigma^2$. (The hint says using
5.5(e), but that is convergence with distribution, which I don't think is correct).
(Update: this is a misprint which is noted in the errata2.pdf for the book).

\bigskip\noindent
%%%%%%%%%%%%%%%%%%%%%%%%%%%%%%%%%%%%%%%%%%%%%%%%%%%%%%%%%%%%%%%%%%%%%%%%%%%%%%%
\textbf{5.2}\\  % PDF page 82
Let $X_1,X_2\ldots$ be a sequence of random variables. Show that $X_n \raQ b$
if and only if
$$
\lim_{n\ra\infty}\E[X_n] = b,
\quad\text{and}\quad
\lim_{n\ra\infty}\V[X_n] = 0.
$$
\textsc{Proof}.\\
$\Rightarrow$) We assume $X_n \raQ b$, which means
$$
\E[(X_n - b)^2]\ra 0,\quad n\ra\infty.
$$
We can rewrite the expression:
\begin{align*}
    \E[(X_n - b)^2] &= \E[X_n^2 - 2bX_n + b^2] \\
    &= \E[X_n^2] - 2b\E[X_n] + b^2 \\
    &= \E[X_n^2] - \E[X_n]^2 + \E[X_n]^2 - 2b\E[X_n] + b^2 \\
    &= \V(X_n) + (\E[X_n] - b)^2
\end{align*}

\newpage\noindent
By our assumption $\E[(X_n - b)^2]\ra 0$ as $n\ra\infty$, and so 
$\V(X_n) + (\E[X_n] - b)^2 \ra 0$ as $n\ra\infty$.
Since $\V(X_n) \geq 0$ and $(\E[X_n] - b)^2 \geq 0$, then we can conclude that
$$
\V(X_n) \ra 0,
\qquad
(\E[X_n] - b)^2 \ra 0 \imp \E[X_n] \ra b
$$
as $n\ra\infty$.\\
$\Leftarrow$) We assume 
$$
\lim_{n\ra\infty}\E[X_n] = b,
\quad\text{and}\quad
\lim_{n\ra\infty}\V[X_n] = 0.
$$
Then $\lim_{n\ra\infty} \V(X_n) + (\E[X_n] - b)^2 = 0$.
By reversing the calculations from the first part:
$$
\lim_{n\ra\infty} \V(X_n) + (\E[X_n] - b)^2 = 0
\imp
\lim_{n\ra\infty} \E[(X_n - b)^2] = 0
\imp
X_n \raQ b.
$$
By showing implication both ways, the result is proved.\qed

\bigskip\noindent
%%%%%%%%%%%%%%%%%%%%%%%%%%%%%%%%%%%%%%%%%%%%%%%%%%%%%%%%%%%%%%%%%%%%%%%%%%%%%%%
\textbf{5.3}\\  % PDF page 82
Let $X_1,\ldots, X_n$ be IID and let $\mu = \E[X_i]$ with finite variance.
Show that $\bar{X} \raQ \mu$.

\medskip\noindent\textsc{Proof}. Define:
$$
\bar{X}_n = \frac{1}{n}\sum_{i=1}^n X_i.
$$
By the WLLN, $\bar{X}_n\raP \mu$, which we can also express as
$$
\lim_{n\ra\infty}\E[\bar{X}_n] = \mu.
$$
The variance is finite, so $\V(X_i) = \sigma^2 <\infty$. This means that for
the sample variance:
$$
\lim_{n\ra\infty} \V(\bar{X}) = \lim_{n\ra\infty} \frac{\sigma^2}{n} = 0.
$$
With this, we can simply apply the result from exercise 5.2 which shows that
$\E[(\bar{X} - \mu)^2]\ra 0$ as $n\ra\infty$ which proves $\bar{X}\raQ \mu$.\qed

\bigskip\noindent
%%%%%%%%%%%%%%%%%%%%%%%%%%%%%%%%%%%%%%%%%%%%%%%%%%%%%%%%%%%%%%%%%%%%%%%%%%%%%%%
\textbf{5.4}\\  % PDF page 82
Let $X_1,X_2,\ldots$ be a sequence of random variables such that
$$
\P\left(X_n = \frac{1}{n}\right) = 1 - \frac{1}{n^2},
\quad\text{and}\quad
\P(X_n = n) = \frac{1}{n^2}.
$$
Just noting the probabilities for each outcome as $n$ grows.
$$
\begin{tabular}{l|l|l|l|l|l}
           &$n=1$&$n=2$&$n=3$&$n=4$&$\ldots$ \\
           \hline
\rule{0pt}{10pt}$p(x) = \P(X_n = \frac{1}{n})$ &$p(1) = 0$&$p(\frac{1}{2}) = \frac{3}{4}$&$p(\frac{1}{3}) = \frac{8}{9}$&$p(\frac{1}{4}) = \frac{15}{16}$&$\ldots$\\
\rule{0pt}{10pt}$p(x) = \P(X_n = n)$ &$p(1) = 1$&$p(2) = \frac{1}{4}$&$p(3) = \frac{1}{9}$&$p(4) = \frac{1}{16}$&$\ldots$\\
\hline
\end{tabular}
$$
As $n$ becomes large, the probability that we get $n$ and not $1/n$ becomes very small.
But as $n$ becomes large, it will also yield extreme outliers in the sequence with
a non-zero probability.

\newpage\noindent
Starting by calculating the mean, second moment, and variance.
\begin{align*}
    \E[X_n] &= \left(\frac{1}{n}\right)\left(1 - \frac{1}{n^2}\right) + (n)\left(\frac{1}{n^2}\right)\\
    &= \frac{1}{n} - \frac{1}{n^3} + \frac{1}{n} \\
    &= \frac{2}{n} - \frac{1}{n^3}
\end{align*}
\begin{align*}
    \E[X_n^2] &= \left(\frac{1}{n^2}\right)\left(1 - \frac{1}{n^2}\right) + (n^2)\left(\frac{1}{n^2}\right)\\
    &= \frac{1}{n^2} - \frac{1}{n^4} + 1
\end{align*}
\begin{align*}
    \V(X_n) &= \E[X_n^2] - \E[X_n]^2 \\
    &= \frac{1}{n^2} - \frac{1}{n^4} + 1 - \left(\frac{2}{n} - \frac{1}{n^3}\right)^2 \\
    &= \frac{1}{n^2} - \frac{1}{n^4} + 1 - \left(\frac{4}{n^2} - \frac{4}{n^4} + \frac{1}{n^6}\right) \\
    &= 1 - \frac{3}{n^2} + \frac{3}{n^4} - \frac{1}{n^6}
\end{align*}
Results verified by simulation. (See code in \texttt{5.4.R}). E.g. for $n=200$ and 10M simulations:
\begin{lstlisting}[style=RSyntax, title=R]
# Simulated vs. Theoretical
> mean(Xn)
[1] 0.009879878
> 2/n - 1/n^3
[1] 0.009999875
> var(Xn)
[1] 0.9759275
> 1 - 3/n^2 + 3/n^4 - 1/n^6
[1] 0.999925
\end{lstlisting}
From these expressions we can see that $\lim_{n\ra\infty} \E[X_n] = 0$, but $\lim_{n\ra\infty} \V(X_n) = 1$.
Since the variance does not become 0, we know by the result in exercise 5.2 that this does
NOT converge in quadratic mean.

Checking if $X_n$ converges in probability. If we fix some $\eps > 0$, we can apply
the Chebyshev inequality. Set $\mu = \E[X_n]$ and $\sigma^2 = \V(X_n)$:
$$
\P(|X_n -\mu| > \eps) \leq \frac{\sigma^2}{\eps^2}
= \frac{1 - \frac{3}{n^2} + \frac{3}{n^4} - \frac{1}{n^6}}{\eps^2}
$$
By taking the limit $n\ra\infty$ on both sides, we get:
$$
\lim_{n\ra\infty} \P(|X_n -\mu| > \eps)
\leq
\lim_{n\ra\infty} \frac{1 - \frac{3}{n^2} + \frac{3}{n^4} - \frac{1}{n^6}}{\eps^2}
= \frac{1}{\eps^2}.
$$
Since this does not tend to 0, we do NOT have convergence in probability.

\newpage\noindent
%%%%%%%%%%%%%%%%%%%%%%%%%%%%%%%%%%%%%%%%%%%%%%%%%%%%%%%%%%%%%%%%%%%%%%%%%%%%%%%
\textbf{5.5}\\  % PDF page 82
Let $X_1,\ldots,X_n\sim\text{Bernoulli}(p)$. Prove that
$$
\frac{1}{n}\sum_{i=1}^n X_i^2 \raP p,
\quad\text{and}\quad
\frac{1}{n}\sum_{i=1}^n X_i^2 \raQ p.
$$
\textsc{Proof}. Recalling how to calculate the expectation and second moment
for a Bernoulli$(p)$ variable:
$$
\E[X] = (1)p + (0)(p-1) = p,
\qquad
\E[X^2] = (1)^2p + (0)^2(p-1) = p,
$$
With these, we can calculate the variance.
$$
\V(X) = \E[X^2] - \E[X]^2 = p - p^2 = p(1 - p).
$$
%Since the $X_i$ are IID, we can apply the WLLN to establish that $\bar{X} \raP p$.
We define $Y_i := X_i^2$ and exploit the simplicity of the Bernoulli distribution.
In this case:
$$
\E[Y] = \E[X^2] = (1)^2p + (0)^2(p-1) = p,
\qquad
\E[Y^2] = \E[X^4] = (1)^4p + (0)^4(p-1) = p,
$$
With these, we can calculate the variance.
$$
\V(Y) = \E[Y^2] - \E[Y]^2 = p - p^2 = p(1 - p).
$$
So we have $\mu = \E[Y] = p$ and $\sigma^2 = p(1-p)$.
We can now define the sample mean and variance:
$$
\E[\bar{Y}] = p,\qquad \V(\bar{Y}) = \frac{\sigma^2}{n} = \frac{p(1 - p).}{n}.
$$
Since the variance tends to 0 as $n\ra\infty$ we can use the results in exercise 5.3
to conclude that $\bar{Y}\raQ p$ and by how we defined $Y$, $\frac{1}{n}\sum_{i=1}^n X_i^2 \raQ p$.
Since we have convergence in quadratic mean, it follows that we also have convergence
in probability. \qed

\bigskip\noindent
%%%%%%%%%%%%%%%%%%%%%%%%%%%%%%%%%%%%%%%%%%%%%%%%%%%%%%%%%%%%%%%%%%%%%%%%%%%%%%%
\textbf{5.6}\\  % PDF page 82
The height of men has mean 68 inches and standard deviation 2.6 inches. We have $n=100$.
Finding the approximate probability that the average height in the sample will
be at least 68 inches.

 We can approximate this probability with the CLT (central limit theorem). We want to
 find the probability that the sample height $\bar{X} = \frac{1}{n}\sum_{i=1}^n X_i$
 is at least as big as the population mean: $\P(\bar{X} > \mu)$. 
 $$
 \P(\bar{X} > \mu) =
 1 - \P(\bar{X} \leq \mu) %\approx 1 - \P(Z_n \leq \mu)
 $$
 By the central limit
 theorem, using that $n=100$, $\mu = 68$ and $\sigma = 2.6$:
 $$
 \P(\bar{X} \leq 68) =
 \P(\bar{X} - 68\leq 0) =
 \P\left(\frac{10(\bar{X} - 68)}{2.6} \leq 0\right) \approx
 \P(Z\leq 0) = 0.5
 $$
 (Since the standard normal distribution is symmetric and centered at 0).
 So we get:
 $$
 \P(\bar{X} > 68) =
 1 - \P(\bar{X}\leq 68) = 1 - 0.5 = 0.5.
 $$

 \newpage\noindent
 %%%%%%%%%%%%%%%%%%%%%%%%%%%%%%%%%%%%%%%%%%%%%%%%%%%%%%%%%%%%%%%%%%%%%%%%%%%%%%%
 \textbf{5.7}\\  % PDF page 82
 Let $\lambda_n = \frac{1}{n}$ for all $n$ and let $X_n\sim\text{Poisson}(\lambda_n)$.

 \medskip\noindent(a) Showing that $X_n \raP 0$. By properties of the Poisson
 distribution, we have
 $$
\mu = \E[X_n] = \frac{1}{n},\qquad \sigma^2 = \V(X_n) = \frac{1}{n}.
 $$
 By fixing some $\eps > 0$ and applying Chebyshev's inequality, we get:
 $$
 \P(|X_n - \frac{1}{n}| > \eps) = \P(|X_n - \mu| > \eps) \leq \frac{\sigma^2}{\eps^2} = \frac{1}{n\eps^2}
 $$
By taking the limit on both sides:
$$
\lim_{n\ra\infty}\P(|X_n - \frac{1}{n}| > \eps) =  \leq \lim_{n\ra\infty}\frac{1}{n\eps^2} = 0,
$$
and since $\mu = 1/n \ra 0$ we have shown that $X_n\raP 0$.

\medskip\noindent(b) We define $Y_n = nX_n$ and will show that $Y_n\raP 0$.
Finding the mean and variance:
$$
\E[Y_n] = \E[nX_n] = n\E[X_n] = n\left(\frac{1}{n}\right) = 1
$$
$$
\V(Y_n) = \V(nX_n) = n^2\V(X_n) = n^2\left(\frac{1}{n}\right) = n
$$
From the Chebyshev inequality we can only really conclude that $Y_n$ does NOT converge to 1,
but we can't use it for determining the asymptotic behavior of $Y_n$ at 0. Instead we will
use Theorem 5.5(f):$X_n \raP 0$ then $g(X_n)\raP g(X)$ . Here, $Y_n$ is a function of $X_n$,
defined as: $Y_n = nX_n$ where $g(x) = nx$, which is a continuous function.
In this case we get that $X_n \raP 0$ implies $nX_n\raP n\cdot 0$ so $Y_n \raP 0$.

\bigskip\noindent
%%%%%%%%%%%%%%%%%%%%%%%%%%%%%%%%%%%%%%%%%%%%%%%%%%%%%%%%%%%%%%%%%%%%%%%%%%%%%%%
\textbf{5.8}\\  % PDF page 82
A program has $n=100$ pages of code. Let $X_i\sim\text{Poisson}(1)$ be iid and denote the
number of errors on page $i$. Let $Y = \sum_{i=1}^n X_i$ denote the total
number of errors. Use the CLT to approximate $\P(Y < 90)$.

\medskip\noindent The mean and variance are $\mu = \E[X_i] = 1$ and $\sigma^2 = \V(X_i) = 1$.
An important observation: the CLT applies to sample means, but in this case we are simply
summing up 100 independent Poisson variables, and so $Y\sim\text{Poisson}(100)$,
i.e. $\E[Y] = 100$ and $\V(Y) = 100$.

Let us define $W = \frac{1}{n}Y$, which means:
$$
W = \frac{1}{n}Y = \frac{1}{n}\sum_{i=1}^n X_i,
$$
then by the CLT, $W\sim N(1, 1/100)$, so $\E[W] = 1$ and $\V(W) = 1/100$.

\newpage\noindent
By going the other way, we can find a normal approximation for $Y$ by using that $Y = nW$
where $n=100$.:
$$
\E[Y] = \E[nW] = n\E[W] = n(1) = 100,
$$
$$
\V(Y) = \V(nW) = n^2\V(W) = (100)^2\left(\frac{1}{100}\right) = 100
$$
The standard deviation is: $\sqrt{\V(Y)} = 10$. Approximating the probability.
$$
\P(Y < 90) = \P\left(\frac{Y - 100}{10} < -9\right) \approx \P(Z\leq -9) \approx 0
$$
Note: this exercise demonstrates that the CLT is just an approximation,
and under certain conditions it's a very poor approximation. In \texttt{5.8.R}
a simulation repeating the conditions were done one million times, and numerically,
the probability that $\P(X < 90)$ turns out to be about 0.1467.
\begin{lstlisting}[style=RSyntax, title=R]
> # Approximating answer numerically
> length(Y)
[1] 1000000
> sum(Y < 90)/length(Y)
[1] 0.146773
\end{lstlisting}





% By the CLT the normal approximation will be $N = (\mu, \sigma^2/n) = N(1, 1/100)$.
% With this we can approximate the probability:
% $$
% \P(Y < 90) = \P\left(Y - 100 < -10\right)
% = \P\left(10(Y - 100) < -100\right)
% %= \P\left(10(Y - 100) < -100\right)
% $$




\begin{comment}

\bigskip\noindent
%%%%%%%%%%%%%%%%%%%%%%%%%%%%%%%%%%%%%%%%%%%%%%%%%%%%%%%%%%%%%%%%%%%%%%%%%%%%%%%
\textbf{5.X}\\  % PDF page 82


\begin{align*}
    A &= B
\end{align*}


\begin{equation*}
    A = B
    \tag*{\qed}
\end{equation*}


\begin{lstlisting}[style=RSyntax, title=R]
# Code
\end{lstlisting}

\begin{verbatim}
# Output
\end{verbatim}






%%%%%%%%%%%%%%%%%%%%%%%%%%%%%%%%%%%%%%%%%%% Minipages x 2
\begin{figure}[H]
    \begin{minipage}{0.5\textwidth}
        % MINIPAGE 1
    \end{minipage}
    \begin{minipage}{0.5\textwidth}
        % MINIPAGE 2
    \end{minipage}
\end{figure}

%%%%%%%%%%%%%%%%%%%%%%%%%%%%%%%%%%%%%%%%%%% Two R images
\begin{figure}[H]
    \begin{minipage}{0.5\textwidth}
    \begin{center}
        \begin{figure}[H]
            \includegraphics[scale=0.7]{IMG1.pdf}
        \end{figure}
    \end{center}
    \end{minipage}
    \begin{minipage}{0.5\textwidth}
    \begin{center}
        \begin{figure}[H]
            \includegraphics[scale=0.7]{IMG2.pdf}
        \end{figure}
    \end{center}
    \end{minipage}
\end{figure}


%%%%%%%%%%%%%%%%%%%%%%%%%%%%%%%%%%%%%%%%%%% Two TikZ images
%%% Tikz Image - side by side
\begin{figure}
    \begin{minipage}[0.5\textwidth]
\begin{tikzpicture}
    \begin{axis}[
        width=\textwidth,
        axis lines = left,
        ymin = -0.002,
        ymax = 2.1,
        xlabel = $z$,
        ylabel = {$f_Z(z)$},
    ]
    %Section 1
    \addplot [
        domain=0:1, 
        samples=10, 
        color=blue,
        style=ultra thick,
    ]
    {2 - 2*x};
    \end{axis}
\end{tikzpicture}
    \end{minipage}
    \begin{minipage}[0.5\textwidth]
\begin{tikzpicture}
    \begin{axis}[
        width=\textwidth,
        axis lines = left,
        ymin = -0.002,
        ymax = 2.1,
        xlabel = $z$,
        ylabel = {$f_Z(z)$},
    ]
    %Section 1
    \addplot [
        domain=0:1, 
        samples=10, 
        color=blue,
        style=ultra thick,
    ]
    {2 - 2*x};
    \end{axis}
\end{tikzpicture}
    \end{minipage}
\end{figure}
    
    
    
\end{comment}